\documentclass[CSHFoundation.tex]{subfiles}
\begin{document}

\chapter{Church}
\centerline{\includegraphics[scale=0.35]{6-Clogo.png}}
\section{Church Constitution}
\subsection{Purpose}

The church is a community for a group of believers who live in close proximity to each other. The ultimate goal of the church is provide the framework necessary for meaningful discipleship to take place. The church collectively seeks to grow individuals spiritually though a connection to God (the Word being a key component), provide emotional/mental help to those who need it, and physical help to those who need it as well.



The Church is the representation of Christ on earth. While the church has often gone astray from her Lord and Maker, she nevertheless remains loved by God. We hold a commitment to the church, but it is even more then that, for we are the church. We hold a commitment to ourselves to be in community with other believers. We believe in being in community with other believers, and we make it our focus to continue that.



In all of our efforts for spiritual growth, the church must be involved. We cannot raise up students who love God, if they have no learned to love God, if they have not learned to love the Bride of Christ. We cannot raise up good kids in our homes, if they have not learned to love God with all of their heart. We may discipline them, but if they do not have Christ, we have given them nothing.


Everything hangs on the church, for it is responsible for keeping the faith.


The Rock of the Church is her husband, Christ Jesus our Lord. As such,

\subsection{Relationship}

\subsubsection{From Him:}

\begin{itemize}
\item What He says Word of God
\item What He has given us
\item Fruit of the Spirit
\item Gifts of the Spirit
\item His Son, and through Him, right relationship with God
\end{itemize}

\subsubsection{Our Response}



\begin{itemize}
\item Our remembrance of Him communion
\item Our public declaration of our allegiance to Him [Baptism]
\item Our Relationship to Him
\item Our Obedience to Him
\item Our Worship of Him
\item Our Giving to Him
\item Our Prayer to Him
\end{itemize}

If any of these aspects are ignored, the church suffers

\subsection{Roles}

\subsubsection{Apostles}

Apostles serve to establish new churches, but going out and proclaiming the truth where no one has yet heard the Gospel. An Apostle views these new churches as new-born children, continuing to care for them until they reach a maturity in the faith.

\subsubsection{Prophets}

Prophets serve to reveal the thoughts of God to the church. Their words must be tested by other prophets, as a safe-guard against heresy.

\subsubsection{Evangelists}

Evangelists serve to bring the Gospel to people in the same location as their church. Once a person has committed their life to the Lord, the evangelist can provide the necessary connection to an church to help this new believer continue to grow as a Christian.

\subsubsection{Elders}

Elders serve to guide and protect the people who belong to the church. As this is a role which has much influence over others, strict guidelines, as outlined in the NT, must be followed without compromise (1 Tim 3, Titus 1).



\subsubsection{Teachers}

Teachers serve to accurately teach the Word of God.



\subsubsection{Deacons}

Deacons serve to serve, helping the church practically so that others can focus on their ministries.



\subsubsection{Priests}

Everyone is considered a priest, and as such, have access to God. No one has exclusive access to God, and cannot specific y how another comes to God. Also, each individual has something to contribute to the body of Christ. (1 Peter 2:9, 1 Cor 12:12)


\subsection{Meetings}

\subsubsection{Functions}

The church at each Tier is responsible for different aspects.

\subsubsection{Tier 1}

Core Groups meet frequently during the week, and thus are best constructed with local people. Communion is celebrated in this group, as they meet over meals, sing songs, and discuss their day.

They are responsible for:

\begin{itemize}
\item Communion
\item Worship
\item Discipleship
\item Encouragement
\item Care
\item Community
\item Intimacy
\item Empowerment
\item Teaching
\end{itemize}

\subsubsection{Tier 2}
\begin{itemize}
\item Communion
\item Worship
\item Empowerment
\item Teaching
\item Reports of what God is doing
\item Giving and Using Monetary Funds
\item Creation + Sustaining Tier 1 groups
\end{itemize}

\subsubsection{Tier 3}
\begin{itemize}
\item Communion
\item Worship
\item Starting and Sustaining projects
\item Dialoguing with Tier 3 homes, church's, and schools.
\item Creation + Sustaining Tier 2 groups
\item Sending Missionaries and Creation of new CSHFoundations (in collaboration with schools and homes)
\end{itemize}

\subsubsection{Leadership in Tiers}

A leader in one tier is a leader in another teir, but that does not exclude persons from being a deacon in Tier 1, 2 and 3 if they are not a leader in Tier 1. Some of these deacon functions include:

\begin{itemize}
\item Deacon of Worship
\item Deacon of Monetary Funds
\item Deacon of Missions
\item Deacon of a particular project
\item Deacon of School
\item Deacon of Home
\end{itemize}

Deacons from multiple Tiers also meet together to collaborate on the various functions they are doing. For example, a Tier 2 Deacon of Monetary Funds will meet together with other Deacons from Tier 2 churchs to make decisions for Tier 3 Monetary Funds.

Each leader is not above each other, and are servants to the flock. They work together, coming to consensus in moving forward in ministry.

\subsection{Serving}



\subsubsection{Ministries}

The church participates in various ministries, from serving the church, to serving the community, being equipped by the APEET; doing the work laid out before them. (Ephesians 2:10, 4:11-12, 1 Corinthians 12:27-31, 1 Peter 4:10). Ministries range across all groups, allowing those called to participate.



\subsubsection{Missions}

The church also sends out apostles to create new assemblies in other locations. Also giving help to other assemblies that may be struggling. (Romans 10:15, 1 Corinthians 16:1-4, John 17:20-23, Proverbs 27:10).


\subsection{Procedures}



\subsubsection{Called to Ministry, Missions}

Those who feel called, or others see as serving in a certain role, will be first mentioned in the core groups. Once the core group affirms the direction informally, it is brought to the Mantle group, which votes unanimously on the person in the new role. If someone does not vote yes, it is then an opportunity to provide constructive criticism on where they could serve, or steps needed before they can serve. Once this decision has been made, the person is commissioned at the crust group, having hands laid upon them by the elders, and then prayed for by the entire assembly. All members in ministry are considered deacons, all members in missions are considered apostles.



\subsubsection{Updates}

For those serving in various capacities, a monthly report must be given, and if not in person, read to the entire assembly.



\subsubsection{Stepping down}

If the a member serving feels this is no longer a direction for them, they will mention this in their monthly report. At this point, the congregation will pray for someone to rise up to fill their place. This replacing will follow the normal procedure.



\subsubsection{Growth}

The church must be called to divide at a certain point, to maintain the internal integrity. At each growth point a group divides in half. In all cases, one must see the transition as a new starting, and not a departing from the center group. There is no provision for charismatic leaders who desire to have a large small group, they must divide, and train new leaders to be put in place. The church is never a one man show, unless that man be Christ dying on the Cross.



\subsubsection{Teaching}

Depending on the differing gifts enabled, so must the gifts be used. If there be more then one teacher, then they must teach. In Tier 2 and 3, there must be a rotation of speakers, teaching topically or energetically as needed. (1 Corinthians 14:26-33, Ephesians 4:1-16, Ecclesiastes 4:13-16, Proverbs 24:6). In the crust groups, teachers teach from their seat (located as part of the great circle), and while he speaks, others listen.



\subsubsection{Bible Reading}

At the start of each and any meeting, a passage of Scripture is read aloud. Core Groups systematically work their way through different books of the Bible, so that no portion of Scripture is neglected.

\subsubsection{Saturation}

There are certain truths that must be established before others can be built upon them.

\begin{enumerate}
\item God exists
\item God has standards
\item Man has broken those standards
\item Eternal damnation is a result of that breach
\item Man cannot restore himself on his own effort
\item Jesus Christ was a existed
\item Jesus Christ was God incarnate
\item Jesus Christ offers forgiveness for broken standards
\item Man must accept Jesus as Savior
\item Man must live according to the Spirit
\end{enumerate}

Each previous stage must be understood before the next can be taught. Society, under the power of the evil one, pushes the opposite direction, taking away points until only the idea of God exists as a folk tale. The Gospel cannot be presented without an understanding of a God that exists with moral standards and a certain damnation that follows for man due to his breaking of those standards. Salvation means nothing until one understands that they indeed need saving.

\subsection{Definitions}

\subsubsection{Trinity}

God the Father, God the Son, and God the Holy Spirit are 3-in-one, yet each part is distinct.



\subsubsection{God}

The almighty Father



\subsubsection{Jesus}

Jesus is our only way to God, and we cannot on our own merit gain entrance to the King of Kings without first accepting Jesus as the Son of God, and making him Lord and Savior of our Life.



\subsubsection{Holy Spirit}

The Holy Spirit indwells every believer who confesses that Jesus Christ is Lord, and through the Spirit God grants different gifts to the members of his body, so that all might be built up.



\subsubsection{Bible}

The Bible is the basis for truth, and anything added to it or taken away from it weakens the body of Christ. Add preaching must be from the Bible.



\subsubsection{Sin}

Man is inherently sinful and doomed to destruction. It is the grace of God that saves man. Without believing in Christ, we cannot be saved. If is not by our own righteousness that we can be saved. No amount of rules and regulations can make us right with God, for we are all sinners at heart. A true belief in Christ entails repentance and and change of heart.


\subsubsection{Church}

The church is a group of believers who have been called out of the world to meet together. A church is defined by the following 4 characteristics: Believers who meet regularly together for the purpose of spiritual growth belonging to a permanent location. The people of an church continue to be an church even when they are not formally meeting together. In all circumstances, it is ideal for believers living in the same general location to consider themselves once church


\section{Education}

It is the role of the church to educate in biblical truth. As such, it works in conjunction with both the school and the home to teach biblical truth to kids.

\subsection{Theology}

\begin{enumerate}
\item Torah
\item Historical
\item Wisdom
\item Projects
\subitem Major Prophets
\subitem Minor Prophets
\item Gospels
\item Acts
\item Paulin Epistles
\item General Epistles
\item Revelation
\end{enumerate}

\end{document}
